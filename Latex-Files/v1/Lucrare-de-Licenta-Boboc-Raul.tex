\documentclass[11pt,a4paper]{report}

\usepackage{fancyhdr}
\usepackage{amssymb}
\usepackage{graphicx}
\usepackage[T1]{fontenc}
\usepackage{hyperref}
\usepackage{epstopdf}
\usepackage{makeidx}
\usepackage{tocloft}

\hypersetup{colorlinks=true, linkcolor=black, citecolor=black, filecolor=black, urlcolor=blue}
\renewcommand*{\cftchapleader}{\cftdotfill{\cftdotsep}}
\renewcommand{\contentsname}{Cuprins}
\renewcommand{\bibname}{B\lowercase{ibliografie}}
\renewcommand{\chaptername}{Capitolul}
\renewcommand{\appendixname}{Anexa}
\renewcommand{\indexname}{I\lowercase{ndice}}

\pagestyle{fancy}
\makeindex

\title{\textit{Instrument de Inspectie Vizuala}}
\author{Boboc Raul Madalin\\
        Facultatea de Informatica\\
        Universitatea
        ,\hspace{-0.02cm},Alexandru Ioan Cuza", Iasi}

\begin{document}
\maketitle

\pagenumbering{roman}
% \include{Pagina_de_titlu}
\tableofcontents
\thispagestyle{empty}

\fancyhf{}
\clearpage
\pagenumbering{arabic}

\chapter*{Introducere}
\addcontentsline{toc}{chapter}{Introducere}
Acesta este un capitol introductiv.
\newpage

\chapter*{Abstract}
\addcontentsline{toc}{chapter}{Abstract}

\chapter{Hardware}
\section{Design Mecanic}
\section{Circuit Electronic}

\chapter{Software}
\section{Arhitectura}
\section{Master}
\section{Slave}

\appendix
\chapter*{Anexa}
\addcontentsline{toc}{chapter}{Anexa}

% \phantomsection
% \addcontentsline{toc}{chapter}{Index}

\begin{thebibliography}{99}
    
\end{thebibliography}
\addcontentsline{toc}{chapter}{Bibliografie}

\end{document}
